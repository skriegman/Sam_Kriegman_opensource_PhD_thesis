% 
\noindent
A robot's mechanical parts routinely wear out from normal functioning and can be lost to injury. For autonomous robots operating in isolated or hostile environments, repair from a human operator is often not possible.
Thus, much work has sought to automate damage recovery in robots.
However, every case reported in the literature to date has accepted the damaged mechanical structure as fixed, and focused on learning new ways to control it.
Here we show for the first time a robot that automatically recovers from unexpected damage by deforming its resting mechanical structure without changing its control policy.
We found that, especially in the case of ``deep insult'', such as removal of all four of the robot's legs, the damaged machine evolves shape changes that not only recover the original level of function (locomotion) as before, but can in fact surpass the original level of performance (speed).
This suggests that shape change, instead of control readaptation, may be a better method to recover function after damage in some cases.
%
