\begin{abstract}
% Abstract length should not exceed 200 words

Typically, AI researchers and roboticists try to realize
intelligent behavior in machines by tuning parameters of a 
predefined structure (body plan and/or neural network
architecture) using evolutionary or learning algorithms. 
Another but not unrelated longstanding property of these systems is their brittleness to slight aberrations, as highlighted by the growing deep learning literature on adversarial examples.
Here we show robustness can be achieved by
evolving the 
geometry of soft robots, their
control systems, and how
their material properties develop
in response to one particular interoceptive stimulus
(engineering stress) during their lifetimes.
By doing so we realized robots that 
were equally fit but more robust to 
extreme material defects (such as 
might occur during fabrication or by damage thereafter)
than robots that did not develop during their lifetimes,
or developed in response to a different interoceptive
stimulus (pressure).
This suggests that the interplay between changes
in the containing systems
of agents (body plan and/or neural architecture)
at different temporal scales (evolutionary
and developmental) along different modalities
(geometry, material properties, synaptic weights)
and in response to different signals (interoceptive
and external perception) all
dictate those agents' abilities to evolve or 
learn capable and robust strategies.
\end{abstract}


%
% The code below should be generated by the tool at
% http://dl.acm.org/ccs.cfm
%
\begin{CCSXML}
<ccs2012>
<concept>
<concept_id>10010147.10010178.10010219.10010222</concept_id>
<concept_desc>Computing methodologies~Mobile agents</concept_desc>
<concept_significance>500</concept_significance>
</concept>
</ccs2012>
\end{CCSXML}

\ccsdesc[500]{Computing methodologies~Mobile agents}


\keywords{Soft robotics.}



