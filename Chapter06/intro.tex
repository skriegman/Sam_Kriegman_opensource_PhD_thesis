\section{Introduction}
\label{sec:introduction}

A major characteristic of life is that three broad time scales are relevant to it: evolution, development and physiological functioning.
% \citep{waddington1957strategy}.
Engineered systems, in marked contrast, often employ an evolutionary or learning algorithm to improve their behavior over time, but rarely employ morphological development; any changes to the physical layout are made in between evaluations \citep{sims1994evolving,lipson2000automatic,cheney2013unshackling}, if they are made at all.

Two notable exceptions are modular robots \citep{zykov2005robotics}, which may reconfigure their bodies by adding and removing discrete structures, and soft robots \citep{shepherd2011multigait}, which may continuously alter the local volumes of different parts of their
bodies while behaving. 
Others \citep{bongard2011morphological} have approximated topological change in rigid bodies by extending outward and angling downward 
appendages using a combination of linear and rotary actuators, thus simulating limb growth.

Several computational but embodied models of \textit{prenatal} development have been reported in the literature \cite{dellaert1996developmental,
% dellaert1994toward,
Eggenberger97,
Bongard01,
% bongard2003evolving,
miller2004evolving,
doursat2009organically}.
As implied,
cellular growth therein occurred prior to any physiological functioning. 
Thus, these studies included change during only
two of the three time scales relevant to life:
evolutionary and behavioral change, but not postnatal developmental
change.

The most common argument in favor of development is that some aspects of the environment are unpredictable, so it is advantageous to leave some decisions up to development rather than specifying them genetically.
Although self evident, it remains to determine which mechanisms of development should be instantiated in robots to realize plastic, adaptive, and useful machines.

Naturally, the performance of an evolved system depends on its capacity for evolutionary improvement: its evolvability.
Development can, under certain conditions, smooth the search space evolution that operates in, thus increasing evolvability.
This process, known as the Baldwin effect \citep{baldwin1896new,dennett2003baldwin}, starts with an advantageous characteristic acquired during the development of individuals, such as the callouses in Fig. \ref{fig:calluses}. 
This can create a new gradient in the evolutionary search space, rewarding descendants that more rapidly manifest the trait during their lifetimes \citep{hinton1987learning,kriegman2017morphological}
and retain it through the remainder of their lifetime \citep{kriegman2017minimal}.
Assuming such mutations exist and can be naturally selected \citep{kriegman2017morphological}, following the gradient requires incrementally reducing development in the manifold of the search space that can express variations on the trait \cite{waddington1942canalization}.


However, fitness landscapes that evolution climbs, and development sometimes smooths, tend not to remain static in realistic settings.
On this vacillating landscape, when the best thing to do does not remain the same, a highly evolvable but non-robust system will need to keep starting over from scratch every time the conditions change.
Computational and engineered systems provide countless examples of systems with nearly perfect performance in a controlled environment, such as a factory, but who turn out to be (often comically) brittle to slight changes in their internal structure, such as damage, or their external environment such as moving on to new terrain or transferal from simulation to reality 
\citep{carlson2005ugvs,
koos2013transferability,
% mccloskey1989catastrophic,
% french1999catastrophic,
goodfellow2013empirical,
szegedy2013intriguing,
% goodfellow2014generative,
athalye2017synthesizing,
% gilmer2018adversarial,
% su2017one,
nguyen2015deep}.

Although generally absent from engineered systems (but see \cite{bongard2006resilient,cully2015robots}), the canonical form of robustness is seen to some extent in all organisms, and it comes from the act of development itself.
For example, a plant that grows according to a fixed program will capture less light than a plant that grows toward sunlight.
But there is another, more subtle form of robustness that we will refer to as `intrinsic robustness' because it is a property of a system's structure rather than of the process by which it may change.

Developmental change produces intrinsically robust systems because they evolved from designs that had to maintain adequate performance along additional dimensions of change \citep{bongard2011morphological,kriegman2017morphological}.
Through morphological development specifically, evolution is compelled to maintain designs that are capable across a series of body plans, with different 
% behavioral repercussions of action, and different 
sensor-motor contingencies; and the ability to tolerate such perturbations can become inherited to some extent in descendants' behaviors \cite{bongard2011morphological} and morphologies \cite{kriegman2017morphological}, even when their developmental flexibility is reduced or completely removed by canalization or fabrication.


And yet, despite the ubiquity of
% (postnatal) 
morphological development in nature, and the adaptive advantages it seemingly confers, there are only a handful of cases reported in the literature in which a simulated robot's mechanical structure was allowed to change while it was behaving 
(e.g.~\citep{ventrella1998designing,
komosinski2003framsticks,
bongard2011morphological,
kriegman2017morphological,
kriegman2017minimal}),
all of which modeled morphological development as a genetically predetermined process: the environment could not influence the way in which development unfolded. 


Assuming that an engineered system is capable of local morphological change in response to environmental signals, it is unclear how it should do so, beyond the examples of morphological plasticity observed in nature. 
Examples include Wolff's law \citep{ruff2006s}---bone grows in response to particular mechanical loading profiles---and Davis' law---soft tissue increases in strength in response to intermittent mechanical demands.
One can envisage other such laws that are not known to occur in biology but could be helpful in a specific artificial system, such as end effectors softening in response to pressure, which might enhance their ability to safely manipulate irregular or delicate objects \cite{brown2010universal}.
Indeed, the genesis of the work presented here is one such anecdotal example given in \citep{corucci2017evolutionary}, where a single robot, subjected to an abrupt doubling in gravity, stiffened its body in reaction to the increased pressure.
However, whether or how it could provide a behavioral advantage, nor whether pressure is the best interoceptive signal to developmentally respond to,
was not investigated.

As a step towards a more adequate picture, we introduce here a simple form of a developmental feedback mechanism:
Genetic systems dictate how organisms develop in response to interoceptive stimuli, and development alters the kinds of interoceptive conditions the organism experiences.
More specifically, at every time step, the proposed model of closed-loop development:
\begin{enumerate}
\item `listens to' load signatures generated from movement; and, in response,
\item modifies the robot's rigidity,
\end{enumerate}
which will change the way it distributes load and generates movement at the next time step.


Optimizing a system that may form a continuum of rigid and soft components---and in which this admixture may change over time---is extremely nonintuitive and underexplored.
Thus, a study of the adaptive properties of such systems---and how they can best be optimized to render useful work---is initiated here.

