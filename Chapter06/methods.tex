\section{Methods}
\label{sec:methods}

We evolved locomotive machines constructed from voxels with heterogeneous stiffness. 
Like many organisms \citep{ruff2006s}, the robot's material stiffness progressively changes in response to mechanical loading incurred as the robot behaves.
This ontogenetic change occurs independently at each voxel according to an evolved local rule.


\subsection{Physical simulation.}

The soft-matter physics engine \textit{Voxelyze} \citep{hiller2014dynamic} is used to calculate the movement of robots resulting from their interaction with a virtual 3D terrestrial environment.
Each robot is simulated for 25 times the length of an expansion/contraction cycle (a total of five seconds). 
The displacement between the starting coordinates and the agent's final center of mass (in the $xy$ plane) is recorded.\footnote{\href{https://github.com/skriegman/2018-gecco}{\textcolor{blue}{\textbf{\texttt{github.com/skriegman/2018-gecco}}}} contains the source code necessary for reproducing the results reported in this paper.}


% For a more general purpose repository see \href{https://github.com/skriegman/evosoro}{\textcolor{blue}{\textbf{\texttt{github.com/skriegman/evosoro}}}}


\subsection{Heavy materials.}

% 1e6 -> 1e9 density
% vol = length^3
% mass = vol * density 
% previously each voxel was a centimeter in length and one gram in mass.
% now each voxel is 1 kg/cm^3 in mass, which is heavier than the heaviest material on earth (22 g/cm^3) so this calculation is incorrect. 

Materials are simulated to have increased mass relative to those used by \cite{hiller2012automatic,cheney2013unshackling,cheney2014electro}.
Constructed from heavier materials, many previously mobile robots become crushed under their own weight and require stiffer material to support locomotion with the same geometry.
However, we also restricted actuation amplitude as materials grow stiffer to better approximate the properties of real materials with different stiffnesses.
This creates an interesting and realistic trade-off: the stiffest material can easily support any body plan but cannot move on its own (like a skeleton without muscle), whereas the softest material can readily elicit forward movement in smaller bodies but cannot support many larger and potentially faster-moving body plans, 
such as those with narrow supporting limbs.
Thus, a robot must carefully balance support with actuation.



\subsection{Quad-CPPN encoding: $\mathbf{\mathbb{C}_1,\mathbb{C}_2,\mathbb{C}_3,\mathbb{C}_4}$
}

Following \cite{cheney2013unshackling}, robot physiology is genetically encoded by a Compositional Pattern Producing Network (CPPN) \citep{stanley2007cppn}, a scale-free mapping that biases search toward symmetrical and regular patterns which are known to facilitate locomotion.


Each point on a 10$\times$10$\times$10 lattice is queried by its cartesian coordinates in 3D space and its radial distance from the lattice center.
An evolved CPPN takes these coordinates as input and returns a single value which is used to set some property of that point in the workspace.
We used four independent CPPNs to separately encode: Geometry, Stiffness, Development, and Actuation.



\subsection{$\mathbf{\mathbb{C}_1}\;$ Geometry.}
The geometry of a robot is specified by a bitstring that indicates whether material is present (1) or absent (0) at each lattice point in the workspace, as dictated by $\mathbb{C}_1$.
The robot's geometric shape is taken to be the largest contiguous collection of present voxels.


\subsection{$\mathbf{\mathbb{C}_2}\;$ Stiffness.}
Young's modulus is often used as an approximate measure of material stiffness.
Robots are typically constructed of materials such as metals and hard plastics that have moduli in the order of $10^9 - 10^{12}$ pascals (Pa), whereas soft robots (and natural organisms) are often composed of materials with moduli in the order of
$10^4 - 10^9$ Pa \citep{rus2015design}.


Here, voxels may have moduli in the range $10^4 - 10^{10}$ Pa.
The robot's congenital stiffness is set at each voxel by $\mathbb{C}_2$, but may be changed by development.



% ``Young's modulus is only defined for homogeneous, prismatic bars that are subject to axial loading and small deformations ($< 0.2\%$ elongation for metals) and thus has limited relevance to soft robots and other soft-matter technologies that have irregular (non-prismatic) shape and undergo large elastic or inelastic deformations. Nonetheless, Young’s modulus is a useful measure for comparing the rigidity of the materials that go into a soft robot.'' - Majidi \citep{majidi2014soft}


\subsection{$\mathbf{\mathbb{C}_3}\;$ Development.}
The robot's stiffness distribution $k$ 
can change progressively during its lifetime $t$ in response to localized engineering stress $\sigma$ or pressure $p$.
The rate of change $\alpha_i$ is specified at the $i^{\text{th}}$ voxel by $\mathbb{C}_3$, with possible values in $0\pm10$.
We compare three developmental variants.
% We compare the following developmental variants.
\begin{align}
% \shortintertext{\textit{\textbf{None:}}} 
&\textit{\textbf{None:}} 
\hspace{2em} &\frac{dk_i}{dt}& \,=\, 0 
\hspace{10em}
\label{eq:none} \\[0.35em]
% \shortintertext{\textit{\textbf{Stress:}}}
&\textit{\textbf{Stress:}} \hspace{3em} &\frac{dk_i}{dt}& \,=\, \alpha_i \cdot \frac{d\sigma_i}{dt} 
\label{eq:stress} \\[0.35em]
% \shortintertext{\textit{\textbf{Pressure:}}}
&\textit{\textbf{Pressure:}} \hspace{3em} &\frac{dk_i}{dt}& \,=\, \alpha_i \cdot \frac{dp_i}{dt} 
\label{eq:pressure}
\end{align}
% \\[1em]
% \intertext{As an additional control, we include ballistic development \citep{kriegman2017minimal}: a linear change from the congenital stiffness distribution $k^{\circ}$ to a final stiffness distribution $k^*$ which is predetermined by a repurposed $\mathbb{C}_3$ (that matches the specs of $\mathbb{C}_2$) and thereby ignores both $\sigma$ and $p$.}
% \shortintertext{\textit{\textbf{Ballistic:}}} 
% \frac{dk_i}{dt}& \,=\, \frac{k_i^*-k_i^{\circ}}{dt}
% \end{align}

\subsection{$\mathbf{\mathbb{C}_4}\;$ Actuation.}
% Following \cite{hiller2012automatic,cheney2013unshackling},
Robots are `controlled by' volumetric actuation: a sinusoidal expansion/contraction of each voxel with a maximum amplitude of 50\% volumetric change.\footnote{The scare quotes are intended to highlight the fact that actuation policies no more dictate the movement of a robot than its geometry.}
However, linear damping $\xi$ is implemented into the system such that the stiffest material does not actuate (Eq. \ref{eq:damp}). 
The phase difference $\phi_i$ of each voxel is determined by $\mathbb{C}_4$, which offsets its oscillation relative to a central pattern generator.

Prior to actuation, each voxel has a resting length of one centimeter.
This length is periodically varying ($f=$ 5 Hz) by approximately 14.5\%
($A\approx$ 0.145 cm), but damped by $\xi$.
% (Which is, without damping, exactly $\{1+A\}^3-1=50\%$ volumetric change.)
The instantaneous length of the $i$\textsuperscript{th} voxel is thus:
\begin{equation}
\label{eq:actuation}
\psi_i(t) = 1+A \cdot \sin(2\pi f t + \phi_i) \cdot \xi(k_i) \, ,
\end{equation}
where:
\begin{equation}
\label{eq:damp}
\xi(k_i) = \frac{k_{\max} - k_i}{k_{\max}-k_{\min}} \, .
\end{equation}



\subsection{Evolution.}

We employed a standard evolutionary algorithm, Age-Fitness-Pareto Optimization \citep{Schmidt2011}, which uses the concept of Pareto dominance and an objective of age (in addition to fitness) intended to promote diversity among candidate designs. 

We performed twenty independent evolutionary trials with different random seeds (1-20); in each trial, a population of 24 robots was evolved for five thousand generations.
Every generation, the population is first doubled by creating modified copies of each individual in the population. 
The age of each individual is then incremented by one.
Next, an additional random individual (with age zero) is injected into the population (which now consists of 49 robots). 
Finally, selection reduces the population down to its original size (24 robots) according to the two objectives of fitness (maximized) and age (minimized).


Mutations add/remove/alter a particular node/link of a CPPN.
They are applied by first selecting which networks to mutate, with the possibility to select all four, and then choosing which operations to apply to each.

% Mutations to the geometry $(\mathbb{C}_1)$ are considered invalid if they would result in creatures composed of less than 100 voxels.
% Mutations to $\mathbb{C}_1$ are reapplied until valid or a maximum of 1500 attempts (this limit was almost never reached).


\subsection{Hypothesis testing.}

We use bootstrapping to construct hypothesis tests.
All $P$-values are reported with Bonferroni correction for multiple (typically three) comparisons.
We adopt the following convention: `${\ast}{\ast}{\ast}$' for ${P<0.001;}$ `${\ast}{\ast}$' for $P<0.01;$ and `${\ast}$' for $P<0.05$.
% It is worth noting that all bootstrapped probabilities reported below were corroborated by two-tailed t-tests.



