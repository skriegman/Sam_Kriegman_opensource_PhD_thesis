\section{Conclusion}

In this paper we introduced a minimal yet embodied model of development in order to isolate the intrinsic effect of morphological change in ontogenetic time,
without the confounding effects of environmental mediation.
Even our simple developmental model naturally provides
a continuum in terms of the magnitude of mutational phenotypic impact,
from the very large (caused by early-in-life developmental mutations) 
to the very small (caused by late-in-life mutations). We predict that,
because of this, such a developmental system will be more evolvable
than an equivalent non-developmental system because the latter lacks
this inherent spectrum in the magnitude of mutational impacts.

We showed that even without any sensory feedback, open-loop development can confer evolvability because it allows evolution to sweep over a much larger range of body plans. Our results suggest that widening the span of the developmental sweep increases the likelihood of stumbling across locally optimal designs otherwise invisible to natural selection, which automatically creates a new selection pressure to canalize development around this good form.
This implies that species with completely blind developmental plasticity tend to evolve faster and more `clearsightedly' than those without it.

Future work will involve closing the developmental feedback loop with as little additional machinery as possible to determine when and how such added complexity increases evolvability.



