
% Feedback loops are ubiquitous in nature.
% Organisms are not born with an immutable structure, unable to change in the face of environmental challenges. 
% On the contrary, organisms are exceedingly flexible, their forms modifying in response to different external conditions, which in turn alter, forming a loop.
% It is in part the ability to close the feedback loop -- to perform constant midcourse correction in response to external signals -- that differentiates life from inanimate objects. 
% Plants exhibit extreme morphological change in response to environmental signals, including leaf growth toward light and root growth toward moisture.
% Animals too, change their shape in response to interactions with the world around them.
% Bones strengthen in response to specific load signatures generated by movement \cite{Ruff06};
% synapses are modulated by rewarded experiences.

% Examples abound, it is often assumed that \textit{the} dominate benefit of development is the ability to do so in a directed fashion. After all, a closed developmental feedback loop explicitly increases adaptability and efficiency compared to open-loop change, which is blind to external stimuli.  
% Longstanding theory also suggests that developmental plasticity 
% plays a crucial role in the origin and diversification of novel traits \cite{Moczekrspb20110971}.
% Others have shown that development can in effect `encode', and thus
% avoid on a much shorter time scale, constraints that would otherwise
% be encountered and suffered by non-developmental systems
% \cite{kouvaris2017evolution}.
% However, despite much work in developmental genetics and comparative embryology, many questions remain unanswered. 
% The way in which development 
% guides evolution, and the extent to which it confers evolvability remain unclear.


% Part of the reason for this confusion stems from analytic, rather than synthetic approaches to understanding development.
% It is exceedingly difficult to observe behavior from 
% without and propose internal mechanisms that may have brought it about.
% Such attempts are further confounded by the fact that 
% many different mechanisms can yield the same behavior.
% As a consequence, it is easy to over estimate the complexity of simple evolutionary, developmental, or behavioral mechanisms \cite{braitenberg1986vehicles}.
% An alternative approach starts from the bottom up, or `learning by building'. We can gain a deeper understanding about the types of mechanisms that \textit{could} produce the developmental behavior we observe in nature by building these systems ourselves.