\section{Introduction}
\label{sec3:introduction}

\begin{figure*}[t]
% \hspace{-0.75cm}
\includegraphics[width=1\textwidth]{Chapter03/img/softbot6000}
% \vspace{-0.5cm}
\caption{\label{fig:videos}   \textbf{The evolutionary history of an Evo-Devo robot.} One of the five phylogenies is broken down into five ontogenies which is in turn shown at five points in its actuation cycle. Voxel color indicates the remaining development. Blue for shrinkage, red for growing, and green for no further change. This robot is featured in video at \href{https://youtu.be/gXf2Chu4L9A}{\color{blue}youtu.be/gXf2Chu4L9A}
}
\end{figure*}



% Development at once constrains and enables evolution since the likelihood of a particular mutation is dictated by the developmental structure.

% Organisms exhibit morphological change over many different time scales.
% They evolve, increasing in complexity from generation to generation.
% They develop over the course of their lifetimes: 
% leaves grow toward light, roots toward moisture; 
% bones are strengthen in response to specific load signatures generated by movement \cite{Ruff06}.
% They move:  predators chase prey; 
% young sunflower plants track the Sun from east to west during the day, resetting at night \cite{Atamian587}.
% They learn: synapses are modulated by rewarded experiences.


Many theories have been proposed as to how development can confer evolvability.
Selfish gene theory \cite{dawkins1982extended} suggests that prenatal development from a single-celled egg is not a superfluous byproduct of evolution, but is instead a critical process that ensures uniformity among genes contained within a single organism and in turn their cooperation towards mutual reproduction. 
Developmental plasticity, the ability of an organism to modify its form in response to environmental conditions, is believed to play a crucial role in the origin and diversification of novel traits \cite{Moczekrspb20110971}.
Others have shown that development can in effect `encode', and thus
avoid on a much shorter time scale, constraints that would otherwise
be encountered and suffered by non-developmental systems
\cite{kouvaris2017evolution}.

Several models that specifically address development of embodied agents have been reported in
the literature.
For example Eggenberger \cite{eggenberger1997evolving} demonstrated how shape could emerge during growth in response to physical
forces acting on the growing entity.
Bongard \cite{Bongard01} adopted models of genetic regulatory networks to demonstrate how evolution could shape the developmental
trajectories of embodied agents. Later, it was shown how such development could lead to a form of self-scaffolding
that smoothed the fitness landscape and thus increased evolvability \cite{bongard2011morphological}.
Miller \cite{miller2004evolving} introduced a developmental model that enabled growing organisms to regrow structure
removed by damage or other environmental stress.

In the spirit of Beer's minimal cognition experiments \cite{beer1996toward}, we introduce here a minimal model of morphological development in embodied agents (figure \ref{fig:illustrations}).
This model strips away some aspects of other developmental
models, such as those that reorganize the genotype to phenotype mapping
\cite{eggenberger1997evolving, Bongard01, kouvaris2017evolution} or allow the agent's environment to influence its development
\cite{hinton1987learning, miller2004evolving}.
We use soft robots as our model agents since they provide many more degrees of developmental freedom compared to rigid bodies, and can in principle reduce human designer bias.
Here, development is monotonic and irreversible, predetermined by genetic code without any sensory feedback from the environment, and is thus 
\textit{ballistic} in nature rather than adaptive.

%nac:  is there anything specific (in the underlying theory and hypotheses) about the fact that this is in embodied agents (as opposed to symbolically in Hinton and Nolan)?  We mention that our work introduces an embodied version of this problem a couple times, but never explicitly say why this is meaningful.  

While biological development occurs along a time axis, it has been implied 
in some developmental models that time provides only an avenue for regularities to form across space, and that only
the resulting fixed form --- its spatial patterns, repetition and symmetry --- are 
necessary for increasing evolvability.
Compositional pattern producing networks (CPPNs,  \cite{stanley2007compositional}) explicitly make this assumption in their abstraction of development which collapses the time line to a single point.
While CPPNs have proven to be an invaluable resource in evolutionary robotics \cite{cheney2013unshackling}, we argue here that discarding time 
may in some cases reduce evolvability
and that there exist fundamental benefits of time itself
in evolving systems.


In this paper, we examine two distinct ways by which ballistic development can increase evolvability.
First, we show how an ontogenetic time scale provides evolution with a 
simple mechanism for inducing mutations with a \textit{range} of magnitude
of phenotypic impact: mutations that occur early in the life time
of an agent have relatively large effects while those that occur
later have smaller effects.
This is important since, according to Fisher's geometric model \cite{fisher1930genetical}, the likelihood a mutation is beneficial is inversely proportional to its magnitude:
Small mutations are less likely to break an \textit{existing} solution.
Larger exploratory mutations, although less likely to be beneficial on average, are more likely to provide an occasional path out of local optima.
Second, we posit that changing ontogenies diversify targets for natural selection to act upon, and that advantageous traits `discovered' by the phenotype during this change can become subject to heritable modification through the `Baldwin Effect' \cite{downing2004development}.

Hinton and Nowlan \cite{hinton1987learning} relied on this second effect when they demonstrated how learning could guide evolution towards a solution to which no evolutionary path led. 
We consider a similar hypothesis with embodied robots and ballistic development, rather than a disembodied bitstring and random search. We demonstrate how open-loop morphological development, without feedback from the environment and without direct communication to the genotype, can similarly alter the search space in which evolution operates making search much easier. 
Hinton \& Nowlan's model of learning was a type of environment-mediated development, in the sense that developmental change stops when the `correct specification' is found, and this information is then used to bias selection towards individuals that find the solution more quickly.
Our work demonstrates that this explicit suppression of development is not necessary; and that completely undirected morphological change is enough to confer evolvability.

