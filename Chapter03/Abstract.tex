\begin{abstract}

Different subsystems of organisms adapt over many time scales,
such as rapid changes in the nervous system (learning),
slower morphological and neurological change over the lifetime 
of the organism (postnatal development), and change over
many generations (evolution). Much work has
focused on instantiating learning or evolution in robots, but relatively
little on development. Although many theories have been
forwarded as to how development can aid evolution, it is difficult
to isolate each such proposed mechanism. Thus, here we introduce
a minimal yet embodied model of development: the body of the
robot changes over its lifetime, yet growth is not influenced
by the environment. We show that even this simple developmental
model confers
evolvability because it allows evolution to sweep over a larger
range of body plans than an equivalent non-developmental system,
and subsequent heterochronic mutations
`lock in' this body plan in more morphologically-static descendants.
Future work will involve gradually complexifying the developmental
model to determine when and how such added complexity increases
evolvability.



\end{abstract}




\keywords{Morphogenesis; Heterochrony; Development; Artificial life; Evolutionary robotics; Soft robotics.}




\begin{CCSXML}
<ccs2012>
<concept>
<concept_id>10010147.10010178.10010216</concept_id>
<concept_desc>Computing methodologies~Philosophical/theoretical foundations of artificial intelligence</concept_desc>
<concept_significance>500</concept_significance>
</concept>
<concept>
<concept_id>10010147.10010178.10010219.10010221</concept_id>
<concept_desc>Computing methodologies~Intelligent agents</concept_desc>
<concept_significance>500</concept_significance>
</concept>
<concept>
<concept_id>10010147.10010178.10010219.10010222</concept_id>
<concept_desc>Computing methodologies~Mobile agents</concept_desc>
<concept_significance>500</concept_significance>
</concept>
</ccs2012>
\end{CCSXML}

% \ccsdesc[500]{Computing methodologies~Philosophical/theoretical foundations of artificial intelligence}
% \ccsdesc[500]{Computing methodologies~Intelligent agents}
\ccsdesc[500]{Computing methodologies~Mobile agents}







