\section{Introduction}
\label{sec:intro}

The simulation-reality gap\footnote{Henceforth, ``the reality gap''|as coined by \citet{jakobi1995noise}.} for rigid-bodied robots is steadily closing.
Computational models of rigid body dynamics can now be regularized and tuned so that control policies optimized in simulation are just as successful when tested on the physical system \cite{bongard2006resilient,hwangbo2019learning}.
The reality gap for soft robots, on the other hand, remains uncharted.
It could be wider than the gap for rigid bodies, or not.
Soft bodies are more challenging to accurately simulate, design, and precisely control.
But, they are also, by definition, more permissive to simulation inaccuracies, design flaws, and control precision: 	
A soft gripper or foot will passively conform to complex objects and terrain, reducing the burden on the simulator to perfectly capture any single, ``true'' surface contact geometry.

Quantifying which soft robot designs, policies and behaviors can be faithfully simulated is critical not only for robotics, but also synthetic approaches to understand functional plasticity of biological systems during development and regeneration.
For both domains, testing candidate hypotheses in reality is expensive, time consuming, and, in some cases, dangerous.
With the recent development of several high-space, many-body, GPU-accelerated 
soft body simulators \cite{holden2019subspace,macklin2019non},
% wang2016descent
sim2real for soft robotics and synthetic biology has become more feasible.
However, because these simulators have yet to be employed to design physical systems, their transferability is currently unknown.

Previous work has demonstrated methods that promote successful sim2real transferal of soft object manipulation but not soft robot behavior.
For example, a rigid-bodied robot arm was successfully trained in simulation to fold towels and drape pieces of cloth over a hanger \cite{matas2018sim}.
However, the reality gap was not quantified beyond a binary success rate for each task.
Additionally, the robot's geometry was fixed and controllers were then optimized for it, whereas in the work reported here, the robot's geometry is part of the solution space.

\citet{hiller2011automatic} evolved the overall geometry and distribution of hard and soft materials in simulation, and transferred the structures and passive dynamics of various cantilever beams.
% (designed to deflect into target shapes under downward load). 
In a separate experiment that included actuating materials, Hiller and Lipson evolved the morphology and behavior of soft robots in simulation, and then built one of the evolved designs physically. 
However, in order to transfer the simulated behavior of this one design, the physical robot needed to be placed in a pressure and vacuum chamber,
whereas the hundreds of soft robot designs built here can be internally pressurized and actuated.

More recently, \citet{kriegman2019automated} subjected a simulated soft robot (composed of elastic voxels) to a series of damage scenarios that removed increasingly more of the robot's structure.
In each scenario, the robot was challenged to recover function (locomotion) by deforming its remnant structure, without changing its predamage control policy.
A pair of recovery strategies, automatically discovered by an evolutionary algorithm in simulation, were transferred to reality (using silicone ``voxels''), but function was not:
The physical system could deform its resting structure as dictated by the recovery strategy, but it could not locomote, before or after damage.
The physical robot was heavy, 
had high friction feet,
and was symmetrically actuated in phase, so it just oscillated in place.

To determine the particular challenges and opportunities of soft robot transferal, it would be useful to greatly scale up the number of design/policy pairs transferred. 
To this end, we present a soft robot design and construction kit based on the silicone voxel modules used in \cite{kriegman2019automated}, but miniaturized to increase stability, simplified to improve reproducibility, and arbitrarily actuated to permit the transferal of locomotion.


Other modular yet rigid-bodied robot design and construction kits exist, such as Molecubes \cite{zykov2007molecubes}.
However, our kit is easier, faster, cheaper, and safer to use.
In short, silicone is molded into hollow voxels, and tubing is attached to supply low pressure actuation from a hand pump,
causing volumetric changes in one or more of the voxels (Figs.~\ref{fig:pressure} and \ref{fig:real}).
For simple behaviors robust to actuation noise,
there is no need to use a highly-pressurized air supply or program microcontrollers for open-loop control.
There are also no expensive motors or power supplies.

Here, we employ the kit as an instrument to measure the reality gap as a function of morphology (Table~\ref{table:lit_review}).
To do so, we fabricated 108 morphologies (transferal of structure) and compared the behavior of nine simulated designs to their silicone equivalents (transferal of behavior).
We hope that the kit's affordability, safety, speed, and simplicity will generate increasingly more, and more reproducible, data about the automated design of increasingly competent soft machines.

