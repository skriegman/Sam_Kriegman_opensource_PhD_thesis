\subsection*{The simulation.}

We used the soft-body physics engine Voxelyze \cite{hiller2014dynamic} to simulate robots composed of actuating and/or passive, elastic voxels.
The simulator models the distance between adjacent voxels as Euler-Bernoulli beams (critically damped; $\zeta=1$).
Additionally, a collision detection system monitors the distance between the voxels on the surface of the robot at each timestep.
If a pair of surface voxels are detected to collide (intersect), a temporary beam (underdamped; $\zeta=0.8$) is constructed between the two until the collision is resolved. 
% (This temporary bond is deleted once the voxels no longer intersect.)

Designs were simulated with a gravitational acceleration of -9.81 m/s$^2$, and initialized on top of an infinite surface plane at $z=0$.
Coulomb friction is applied to voxels in contact with the surface plane.
% Slow (global?) damping coefficient of 0.01 was used.
Voxels were simulated to have 1 cm$^3$ resting volume (resting beam lengths), with Young's modulus $10^7$ Pa, Poisson's ratio 0.35, 
% density $10^6$ kg/m$^3$, % todo: this cannot be right
and coefficients of static and kinetic friction of 1 and 0.5, respectively.
These hyperparameters were adopted from \cite{kriegman2019automated}.
For more details about how the physics are actually modeled, see \cite{hiller2014dynamic}.

Volumetric actuation was implemented by varying the rest length between voxels, in all three dimensions, when computing the elastic force between them.
Volumetric expansion in simulation and reality are both roughly spherical
(Fig.~\ref{fig:pressure}b), but compression in reality is more complex
and difficult to simulate: 
the voxels buckle (Fig.~\ref{fig:pressure}c).
So volumetric actuation was here limited to expansion only ($+90\%$ rest volume).
The active voxels expand in phase with each other 
as dictated by
a central pattern generator:
a sine wave with frequency 4 Hz and amplitude 1.9 cm$^3$.
When the sine wave is at or below zero, the active voxels remain at their resting volume (1 cm$^3$).
This produced quasistatic dynamics.

Each design was simulated for 8831 timesteps, with a stepsize of 0.000453 seconds, resulting in a total simulation time of 4 seconds.
During the first 552 times steps (0.25 sec), the design was allowed to settle under gravity before actuation begins, ensuring that movement (if any) is a result of actuation, rather than passively falling forward.
Just before actuation, the design's initial center of mass is recorded as $(x_0, y_0, z_0)$.
The active voxels are then actuated for 3.75 sec at 4 Hz, or 15 actuation cycles.

An exhaustive search of all 6561 designs (in batches of 50) took 58 CPU hours (1.8 wall-clock hours) on a single AMD Ryzen threadripper 1950X 16-core/32-thread processor.
Fitness was taken to be the net displacement (away from the origin in any direction) of the design's center of mass, in terms of euclidean distance in the plane,
where the origin is defined by the $x,y$ components of the design's initial center of mass $(x_0, y_0)$.
Fitness is thus defined as:
\begin{equation}
    \label{eq:fitness}
    F=\sqrt{(x_t-x_0)^2+(y_t-y_0)^2} \,,
\end{equation}
where $x_t,y_t$ are the final coordinates of the design at the end of the evaluation period.
